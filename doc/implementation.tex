\section{Implementation}
\subsection{Introduction}
In a previous version of the query language, code was generated for a DSM system
specified in \cite{285:Fischer2010}. While the introduced system had its 
limitations, extensions for this system were defined in \cite{297:Frey2010}. In 
particular, this included a XML schema for the code generator, operators in the 
framework and a message format for the DSM framework. This work provides a code 
generator for LUA which forms the foundation for the rewritten DSM framework for 
the DataStreamMiner project. It is a complete rewrite of the code generator of 
previous versions of the query language.

The remainder of this section is organized as follows. First, building blocks of 
the query language such as streams, barriers and operators and their 
corresponding LUA counterpart are presented. Second, deprecated features of the 
query language are discussed followed by the integration of knowledge 
representations into the query language. The section concludes with a short
summary.
\begin{center}
  \begin{tabular}{|l|l|l|l|}
  \hline 
  \textbf{Operator Flow} &  \textbf{Operator LUA} &  \textbf{Comment} \\
  \hline 
  \hline 
  \texttt{avg} & IntAvg, \textit{FloatAvg} & computes the average value \\
  \hline 
  \texttt{count} & MessageCounter & counts a element in a stream \\
  \hline 
  \texttt{std} & \textit{StandardDeviation} & computes standard deviation of a element \\
  \hline 
  \texttt{add} & \textit{Math} & addition operation \\
  \hline 
  \texttt{sub} & \textit{Math} & subtraction operation \\
  \hline 
  \texttt{mult} & \textit{Math} & multiplication operation \\
  \hline 
  \texttt{div} & \textit{Math} & division operation \\
  \hline 
  \texttt{mod} & \textit{Math} & modulo operation \\
  \hline 
  \texttt{join} & Merge & joins multiple operators \\
  \hline 
  \texttt{ejoin} & \textit{ElementMerge} & adds elements to the stream \\
  \hline
  \texttt{split} & \textit{Split} & duplicates a stream \\
  \hline
  \texttt{filter} & \textit{Filter} & filter a stream after a multiple criterias \\
  \hline
  \texttt{tag} & \textit{Tag} & map elements stream/ontology \\
  \hline   
  \texttt{untag} & \textit{Untag} & map elements ontology/stream \\
  \hline 
  \texttt{swrl} &  \textit{SWRL} & operator for specification of SWRL rules \\
  \hline
  \texttt{dtree} & DecisionTree & decision tree \\
  \hline  
  \texttt{sparql} & Tag (Sparql) & specifies sparql queries \\
  \hline 
  \texttt{in} & CacheIn & input operator \\
  \hline
  \texttt{out} & CacheOut & output operator \\
  \hline
  \texttt{rand} & RandomInts & generates random integer values \\
  \hline
  - & PrintOnScreen & prints data on screen \\
  \hline  
  - & SeparateOddEven & operator for modulo two \\
  \hline  
  - & NumberGenerator & operator which generates numbers \\
  \hline  
  \texttt{log} & Log & log operator \\
  \hline
  \end{tabular}
  \captionof{table}{\emph{Comparision of operators in query language and framework}}
  \label{tab:overview}
\end{center}

\subsection{Language}
Table \ref{tab:overview} depicts a comparison of operators in the query language
and the corresponding name of the operator in LUA. Operators which are italic 
are not yet implemented in the DataStreamMiner framework, but will be generated
by the generator of the query language.

\subsubsection{Streams}
While streams are the main entities of the query language, the DSM framework 
itself focuses on operators and the interconnection between them. Thus, streams
are set as parameters of an operator where a additional string denotes if it is 
a input or a output stream. Listing \ref{lst:stream} depicts an example where 
two streams \texttt{one} and \texttt{two} are connected to the input and output 
of an operator.
\begin{lstlisting}[language=Lua, caption={\emph{Example of Streams in LUA}},label={lst:stream}]
-- the strings represent the direction
fm:connect_operators(two, "out", one, "in");
\end{lstlisting}

\subsubsection{Barriers}
The query language introduced two types of barrier mechanisms 
\cite{297:Frey2010}. The first barrier mechanism provides a sliding window over
a stream of data, whereas the sliding window can be specified in terms of time
(seconds, minutes or hours) or elements. The second barrier mechanism is a 
marker mechanism which provides a special element in a data stream. Elements 
in a stream are processed while a marker passes the operator. For further 
information about the marker barrier mechanism please be referred to 
\cite{297:Frey2010}. Since the DSM framework does not support markers, the 
feature was disabled and might be included in future versions of the language 
and the DSM framework. Listing \ref{lst:exbaflw} shows an example for a barrier
in the average operator \texttt{avg} of the query language.
\begin{lstlisting}[language=Flow, caption={\emph{Example of Barrier Mechanism in Flow}},label={lst:exbaflw}]
// Operation: Computes the average of an element 'element' in a stream r over the last 10 hours
a = avg(r.element, r[last 10 hrs]);
\end{lstlisting}
The barrier is transformed into the LUA construct depicted in listing 
\ref{lst:exbalua}. The barrier mechanism is transformed into a 
\texttt{set\_parameter()} operation of the operator. The first parameter in line
one denotes the step size of the sliding window whereas the time is set in 
miliseconds. The last parameter in line 2 sets the type of the sliding window,
here it is \texttt{time}. Another option for the window type is 
\texttt{elements}. At present the \texttt{time} option only supports seconds,
minutes and hours. 
\begin{lstlisting}[language=Lua, caption={\emph{Example of Barrier Mechanism in LUA}},label={lst:exbalua}]
-- step_size_num denotes the step size, window_type the type of the window
a:set_parameter("step_size_num", "36000000");
a:set_parameter("window_type", "time");
\end{lstlisting}

\subsubsection{Operators}
The following section presents examples for operators written in the query 
language and the corresponding  transformation into LUA artifacts. Every 
operator is initialized by a \texttt{create\_operator\_of\_type}() operation. 
Paramters of each operator are set with a \texttt{set\_param\-eter()} operation 
which consists of a key/value structure. The first argument of the operation
denotes the key, the second argument the value of the key. Typically input and
output streams of different operators are connected by the 
\texttt{connect\_operators()} operation. Parameters of the operation are the 
input and output streams and string value which denotes the direction of the 
stream. 

Analysis operators are average (\texttt{avg}), standard deviation (\texttt{std})
and count (\texttt{count}). Each operator has a parameter which denotes the 
element in a stream while the second parameter defines the sliding window over
the stream. Listing \ref{lst:exanflw} depicts examples for the analysis 
operators. The initialization of the streams is omitted. 
\begin{lstlisting}[language=Flow, caption={\emph{Examples of Analysis Operators in Flow}},label={lst:exanflw}]
// Operation: Computes the average of an element 'element' in a stream r over the last 10 hours
a = avg(r.element, r[last 10 hrs]);
// Operation: Computes the standard deviation of an element 'element' in a stream s over the last 20 elements
b = std(s.element, s[last 20 elements]);
// Operation: Counts the occurrence of element 'element' in a stream t in the last 10 minutes
c = count(t.element, t[last 10 min]);
\end{lstlisting}
The operators specified in the previous listing are transformed in the 
corresponding LUA language artifacts which are depicted in listing 
\ref{lst:exanlua}. Each 
\begin{lstlisting}[language=Lua, caption={\emph{Examples of Analysis Operators in LUA}},label={lst:exanlua}]
-- average operator
a = fm:create_operator_of_type("a", "IntAvg");
a:set_parameter("step_size_num", "36000000");
a:set_parameter("window_type", "time");
a:set_parameter("element", "element");
fm:connect_operators(a, "out", r, "in");

-- standard deviation operator
b = fm:create_operator_of_type("b", "StandardDeviation");
b:set_parameter("step_size_num", "20");
b:set_parameter("window_type", "elements");
b:set_parameter("element", "element");
fm:connect_operators(b, "out", s, "in");

-- count operator
c = fm:create_operator_of_type("c", "MessageCounter");
c:set_parameter("step_size_num", "600000");
c:set_parameter("window_type", "time");
c:set_parameter("element", "element");
fm:connect_operators(c, "out", t, "in");
\end{lstlisting}
Arithmetic operators are addition (\texttt{add}), subtraction 
(\texttt{sub}), multiplication (\texttt{mult}), division (\texttt{div}) and
modulo (\texttt{mod}). Parameter of each arithmetic operator is an element in
a stream which represents the left operand of the arithmetic operation and
a right operand which can be a number, variable or another element in a stream. 
The result of each arithmetic operation is a value which is saved in a output
stream. Listing \ref{lst:exarflw} depicts an example for each arithmetic 
operator.
\begin{lstlisting}[language=Flow, caption={\emph{Examples of Arithmetic Operators in Flow}},label={lst:exarflw}]
// Operation: d = u.element + 2
d = add(u.element, 2);
// Operation: e = v.element - o
e = sub(v.element, o);
// Operation: f = w.element * x.element
f = mult(w.element, x.element);
// Operation: g = y.element / 3
g = div(y.element, 3);
// Operation: h = a.element % z.element
h = mod(a.element, z.element); 
\end{lstlisting}
The previously in listing \ref{lst:exarflw} presented examples for 
arithmetic operators are transformed into LUA artifacts and are shown in listing
\ref{lst:exanlua}.
\begin{lstlisting}[language=LUA, caption={\emph{Examples of Arithmetic Operators in LUA}},label={lst:exanlua}]
-- Add
d = fm:create_operator_of_type("d", "Math");
d:set_parameter("operationType", "add");
d:set_parameter("literal", "2");
fm:connect_operators(d, "out", u, "in");
-- Sub
e = fm:create_operator_of_type("e", "Math");
e:set_parameter("operationType", "sub");
e:set_parameter("literal", "42");
fm:connect_operators(e, "out", v, "in");
-- Mult
f = fm:create_operator_of_type("f", "Math");
f:set_parameter("operationType", "mult");
f:set_parameter("stream", "x");
f:set_parameter("element", "element");
fm:connect_operators(f, "out", w, "in0");
fm:connect_operators(f, "out", x, "in1");
-- Div
g = fm:create_operator_of_type("g", "Math");
g:set_parameter("operationType", "div");
g:set_parameter("literal", "3");
fm:connect_operators(g, "out", y, "in");
-- Mod
h = fm:create_operator_of_type("h", "Math");
h:set_parameter("operationType", "mod");
h:set_parameter("stream", "z");
h:set_parameter("element", "element");
fm:connect_operators(h, "out", a, "in0");
fm:connect_operators(h, "out", z, "in1");
\end{lstlisting}
Structural operators are
\begin{lstlisting}[language=Flow, caption={\emph{Examples of Structural Operators}},label={lst:exsysflw}]
// merges the two streams a and b
c = join(a, b);
// adds two elements to a stream b
d = ejoin(newElement, a.element, b);	
// filters stream on the stated condition
e = filter(d.element > 2, d);
\end{lstlisting}
Bla $\wedge$
\begin{lstlisting}[language=Lua, caption={\emph{Generated LUA code of Structural Operators}},label={lst:exanlua}]
-- merge (join) operator
c = fm:create_operator_of_type("c", "Merge");
fm:connect_operators(c, "out", a, "in0");
fm:connect_operators(c, "out", b, "in1");
-- element join operation 
d = fm:create_operator_of_type("d", "ElementMerge");
d:set_parameter("name", "newElement");
d:set_parameter("type", "number");
d:set_parameter("value", "2");
d:set_parameter("stream_name", "a");
d:set_parameter("stream_element", "element");
d:set_parameter("stream_element_type", "int");
fm:connect_operators(d, "out", a, "in0");
fm:connect_operators(d, "out", b, "in1");
-- filter operation 
e = fm:create_operator_of_type("e", "Filter");
e:set_parameter("type", ">");
e:set_parameter("operand", "right");
e:set_parameter("type", "nuber");
e:set_parameter("type", "2");
e:set_parameter("operand", "left");
e:set_parameter("stream", "d");
e:set_parameter("element", "element");
fm:connect_operators(e, "out", d, "in");
\end{lstlisting}
System operatores are the input (\texttt{in}), output (\texttt{out}) and log
(\texttt{log}) operator. The input and ouput operators require a single 
internationalized resource identifier (IRI) or a list of IRIs, a address and a 
port. In addition, the output operator requires a input stream. The log operator
requires a location, a format and a input stream. Examples for the system 
operators are shown in listing \ref{lst:exsysflw}.
\begin{lstlisting}[language=Flow, caption={\emph{Examples of System Operators}},label={lst:exsysflw}]

// The location is a file called 'generated3.txt', the format is the file 'aal', input stream is 'dt'
logIn = log("generated3.txt", "aal", dt);
\end{lstlisting}

\begin{lstlisting}[language=Lua, caption={\emph{Generated LUA code of System Operators}},label={lst:exanlua}]
-- input/cachein operator
cacheIn = fm:create_operator_of_type("cacheIn", "CacheIn");
cacheIn:set_parameter("list_iris", "http://www.anothertest.de ");
cacheIn:set_parameter("connector_socket_address", "localhost");
cacheIn:set_parameter("connector_socket_port", "9595");
-- log operator
logIn = fm:create_operator_of_type("logIn", "Log");
logIn:set_parameter("log_file_location", "generated3.txt");
logIn:set_parameter("log_format", "aal");
fm:connect_operators(logIn, "out", dt, "in");
\end{lstlisting}

\subsubsection{Deprecated Features of the Query Language}
Operators and language features which were part of previous version of the query
language are denoted in table \ref{tab:deprecated}. The expressivness of a query
language is limited to the system on which the language is executed.

The transformation of procedural statements to a complex event processing system
has its limitations. Typically, a complex event processing system can be 
intepreted as a acyclic directed graph, where edges can be intepreted as 
operators and vertices as streams. Control structures such as loops violate the
acyclic property of a acyclic directed graph. Conditional structurs such as 
if/else or switch statements can be expressed by means of the \texttt{filter}
operator and therefore are syntactic sugar. 
\begin{center}
  \begin{tabular}{|l|l|l|}
  \hline
  \textbf{Feature} & \textbf{Type} & \textbf{Comment} \\
  \hline
  \hline
  \texttt{map} & Operator & Operator for projections \\
  \hline
  marker & Operator & Mechanism for creating flexible barriers \\
  \hline
  \texttt{match} & Operator & Operator which compared two streams depending on a expression \\
  \hline
  \texttt{dif} & Operator & Operator for creating differences in sets (set theory) \\
  \hline
  \texttt{sdif} & Operator & Operator for creating symmetric differences in sets (set theory) \\
  \hline
  loops & Feature & Loop statements such as while  \\
  \hline
  conditions & Feature & Conditional statements such as if/else or switch/case \\
  \hline
  functions & Feature & Procedural functions for the map operator \\
  \hline
  \end{tabular}
  \captionof{table}{\emph{Deprecated features of the query language}}
  \label{tab:deprecated}
\end{center}

\subsection{Integration of Meta-models}
Xtext allows to reference instances of metamodels within a DSL. One example is 
the use of OWL classes in the query language. In order to use this mechanism 
several steps are required. First, the metamodel must be referenced in the 
grammar of the language
\begin{lstlisting}[language=, caption={\emph{bla}},label={lst:shell}]
...
import "platform:/resource/de.hs_rm.cs.vs.dsm.flow/model/owl.ecore"
   as owl
...
\end{lstlisting}
Second, the metamodel needs to be referecend in the MWE2 workflow of the 
language project.
\begin{lstlisting}[language=C, caption={\emph{bla}},label={lst:shell}]
...
import "platform:/resource/de.hs_rm.cs.vs.dsm.flow/model/owl.ecore"
   as owl
...
\end{lstlisting}

\subsubsection{Unified Modeling Language}
\subsubsection{Web Ontology Language}
\subsection{Generator}
\subsection{Summary}
