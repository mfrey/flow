%\lstloadlanguages{Flow}
\section{Implementation}
\subsection{Introduction}
In a previous version of the query language, code was generated for a DSM system
specified in \cite{285:Fischer2010}. Required extensions for this system were 
defined in \cite{297:Frey2010}. In particular, this included XML schema for the 
code generator, operators in the framework and a message format for the complex 
event processing system. This work provides a code generator for LUA which forms
the foundation for a rewritten DSM framework for the DataStreamMiner project.

\subsection{Deprecated Features of the Query Language}
Operators and language features which were part of previous version of the query
language are denoted in table \ref{tab:deprecated}. The expressivness of a query
language is limited to the system on which the language is executed.

The transformation of procedural statements to a complex event processing system
has its limitations. Typically, a complex event processing system can be 
intepreted as a acyclic directed graph, where edges can be intepreted as 
operators and vertices as streams. Control structures such as loops violate the
acyclic property of a acyclic directed graph. Conditional structurs such as 
if/else or switch statements can be expressed by means of the \texttt{filter}
operator and therefore are syntactic sugar. 
\begin{center}
  \begin{tabular}{|l|l|l|}
  \hline
  \textbf{Feature} & \textbf{Type} & \textbf{Comment} \\
  \hline
  \hline
  \texttt{map} & Operator & Operator for projections \\
  \hline
  marker & Operator & Mechanism for creating flexible barriers \\
  \hline
  \texttt{match} & Operator & Operator which compared two streams depending on a expression \\
  \hline
  \texttt{dif} & Operator & Operator for creating differences in sets (set theory) \\
  \hline
  \texttt{sdif} & Operator & Operator for creating symmetric differences in sets (set theory) \\
  \hline
  loops & Feature & Loop statements such as while  \\
  \hline
  conditions & Feature & Conditional statements such as if/else or switch/case \\
  \hline
  functions & Feature & Procedural functions for the map operator \\
  \hline
  \end{tabular}
  \captionof{table}{Deprecated features of the query language}
  \label{tab:deprecated}
\end{center}

\subsection{Language}
Table \ref{tab:overview} depicts a comparison of operators in the query language
and the corresponding name of the operator in LUA. Operators which are italic 
are not yet implemented in the DataStreamMiner framework, but will be generated
by the generator of the query language.
\begin{center}
  \begin{tabular}{|l|l|l|l|}
  \hline 
  \textbf{Operator Flow} &  \textbf{Operator LUA} &  \textbf{Comment} \\
  \hline 
  \hline 
  avg & IntAvg, \textit{FloatAvg} & c \\
  \hline 
  count & MessageCounter & c \\
  \hline 
  std & \textit{StandardDeviation} & c \\
  \hline 
  add & \textit{Math} & c \\
  \hline 
  sub & \textit{Math} & c \\
  \hline 
  mult & \textit{Math} & c \\
  \hline 
  div & \textit{Math} & c \\
  \hline 
  join & Merge & c \\
  \hline 
  ejoin & \textit{ElementMerge} & c \\
  \hline
  split & ? & c \\
  \hline
  filter & b & c \\
  \hline
  tag & b & c \\
  \hline   
  untag & b & c \\
  \hline 
  swrl & b & c \\
  \hline
  dtree & DecisionTree & c \\
  \hline  
  a & Tag (Sparql) & c \\
  \hline 
  in & CacheIn & c \\
  \hline
  out & CacheOut & c \\
  \hline
  rand & RandomInts & c \\
  \hline
  ? & PrintOnScreen & c \\
  \hline  
  ? & SeparateOddEven & c \\
  \hline  
  a & NumberGenerator & c \\
  \hline  
  log & Log & c \\
  \hline
  \end{tabular}
  \captionof{table}{Comparision of operators in query language and framework}
  \label{tab:overview}
\end{center}

\subsubsection{Streams}
\subsubsection{Barriers}
\subsubsection{Operators}

\begin{lstlisting}[language=Flow, caption={\emph{Examples of Analysis Operators in Flow}},label={lst:exanflw}]
// Examples of Operators average, standard deviation and count
a = avg(r.element, r[now]);
b = std(t.element, s[last 20 elements]);
c = count(t.element, t[last 10 min]);
// Examples of Operators addition, subtraction, multiplication and division
d = add(u.element, 2);
e = sub(v.element, 4);
f = mult(w.element, x.element);
g = div(y.element, 3);
\end{lstlisting}

\begin{lstlisting}[language=Flow, caption={\emph{Examples of Analysis Operators in LUA}},label={lst:exanlua}]
todo
\end{lstlisting}


\subsection{Integration of Meta-models}
Xtext allows to reference instances of metamodels within a DSL. One example is 
the use of OWL classes in the query language. In order to use this mechanism 
several steps are required. First, the metamodel must be referenced in the 
grammar of the language
\begin{lstlisting}[language=, caption={\emph{bla}},label={lst:shell}]
...
import "platform:/resource/de.hs_rm.cs.vs.dsm.flow/model/owl.ecore"
   as owl
...
\end{lstlisting}
Second, the metamodel needs to be referecend in the MWE2 workflow of the 
language project.
\begin{lstlisting}[language=C, caption={\emph{bla}},label={lst:shell}]
...
import "platform:/resource/de.hs_rm.cs.vs.dsm.flow/model/owl.ecore"
   as owl
...
\end{lstlisting}

\subsubsection{Unified Modeling Language}
\subsubsection{Web Ontology Language}
\subsection{Generator}
\subsection{Summary}
