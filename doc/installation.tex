\section{Installation}
The project uses Maven/Tycho for test, build and deployment of the Eclipse 
application. For the installation of Maven/Tycho please be referred to the 
corresponding documentation \cite{tycho} and \cite{maven}. Please note that 
Tycho requires Maven 3.

\subsection{Development}
\subsubsection{Prerequisites}
The following plugins are required in order to do further development on the 
language and its workbench: 
\begin{itemize}
  % TODO: Herausfinden wie der plugin provider heisst
  \item null (todo)
  \begin{itemize}
    \item Eclipse Modeling Tools
  \end{itemize}
  \item Eclipse Modeling Project
  \begin{itemize}
    \item EMF Compare UML2 Integration
    \item MWE 2 language SDK
    \item MWE 2 runtime SDK
    \item MWE SDK
    \item OCL Examples and Editors
    \item UML2 Extender SDK
    \item Xpand SDK
    \item Xtext SDK
  \end{itemize}
  \item Sonatype
  \begin{itemize}
    \item Tycho Project Configurators
  \end{itemize}
  \item m2e
  \begin{itemize}
    \item m2e Maven Integration for Eclipse
  \end{itemize}
\end{itemize} 
All plugins should be available on the standard eclipse update site, exceptions
are tycho and the maven integration for eclipse. The availability of the 
external plugins is subject to change. For example the tycho project 
configurators are a eclipse incubator project and thus become some day available
on a official eclipse update site. Bullets denote the provider of the plugins
while dashes name the plugin.

In addition, the eclipse delta pack needs to be installed on an existing eclipse
installation in order to build the project via maven/tycho.
 
\subsubsection{Import of the Project}
The plugins of the project are imported via the import functionality of eclipse.
Please select from the \textbf{"File"} menu the entry \textbf{"Import ..."} > 
\textbf{"Maven"} > \textbf{"Existing Maven Projects"}. Select as root
directory the root of the project directory which contains the parent pom.xml. 
Select all projects for an import to your workspace. Usually the import process
takes some time, especially in cases of a first initial import. 

\subsection{Rich Client Application}
The build process is initialised with the following command, where the target 
platform points to the corresponding eclipse (including the eclipse delta pack) 
installation.
\begin{lstlisting}[language=C, caption={\emph{bla}},label={lst:shell}]
frey ~/Projekte $ mvn package -Dtycho.targetPlatform=/home/michael/Software/eclipse
\end{lstlisting}
Tests are executed as usual with maven.
\begin{lstlisting}[language=C, caption={\emph{bla}},label={lst:shell}]
frey ~/Projekte $ mvn test
\end{lstlisting}
The documentation is shipped with builds for both Windows and Linux and features
a 32 and 64 bit version. The javadoc documentation can be generated with the 
following maven target
\begin{lstlisting}[language=C, caption={\emph{bla}},label={lst:shell}]
frey ~/Projekte $ mvn javadoc:javadocs
\end{lstlisting}
At present the project uses the doclava project for documentation.


