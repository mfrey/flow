\section{Introduction}
The project is based on the language Flow \cite{297:Frey2010}. Flow is a 
query/analysis language for the DataStreamMiner framework developed at the 
Distributed Systems Lab (Dopsy) at RheinMain University of Applied Sciences. In 
contrast to the previous version of the project, the language is built using 
Xtext 2.0 as a language workbench toolkit and uses maven/tycho for the build 
process. The target platform of the project is eclipse in version 3.7. The project is split 
in several plugins and is structured as follows:
\begin{center}
  \begin{tabular}{|l|l|}
  \hline
  \textbf{Name} & \textbf{Description} \\
  \hline
  \hline
  \texttt{de.hs\_rm.cs.vs.dsm.flow} & Specifies the grammar of the language and provides a generator for 
  the language \\
  \hline
  \texttt{de.hs\_rm.cs.vs.dsm.flow.tests} &  Unit tests related to the language are defined within the plug-in \\
  \hline
  \texttt{de.hs\_rm.cs.vs.dsm.flow.ui} &  Within the plug-in different user-interface related classes and interfaces are
  specified. \\
  \hline
  \texttt{de.hs\_rm.cs.vs.dsm.flow.feature} & Plug-in contains a feature.xml file which lists the language plug-ins \\
  \hline
  \texttt{de.hs\_rm.cs.vs.dsm.flow.product} & Plug-in specifies the product definition for the RCP application\\
  \hline
  \texttt{de.hs\_rm.cs.vs.dsm.flow.product.feature} & Specifies dependencies to the language plug-ins in a feature.xml\\
  \hline
  \texttt{de.hs\_rm.cs.vs.dsm.owl} & Plug-in consists of classes and interfaces generated from a RDFS and OWL 
  meta model.\\
  \hline
  \texttt{de.hs\_rm.cs.vs.dsm.owl.edit} & Plug-in contains factory classes for creating OWL and RDFS 
  elements \\
  \hline
  \texttt{de.hs\_rm.cs.vs.dsm.owl.editor} & Provides an editor for reading OWL files \\
  \hline
  \texttt{de.hs\_rm.cs.vs.dsm.owl.ui} & Specifies extension points for loading the OWL 
  editor within the language editor. \\
  \hline
  \texttt{de.hs\_rm.cs.vs.dsm.flow.target\_definition} & Specifies the target definition of the RCP application\\
  \end{tabular}
  \captionof{table}{\emph{Plug-ins of the Language and its Workbench}}
  \label{tab:plug-ins}
\end{center}

\subsection{Structure and Organization}
Code and references to classes or other entities of the query language or the 
code generator are set in a different \texttt{font}.

The remainder of the documentation is organized as follows. The process 
of installing the query language framework and how to set up a development
environment for the language is presented in chapter \ref{chapter:installation}.
Chapter \ref{chapter:design} shows details about the design of the LUA code 
generator. Details about the query language and the generated LUA code are 
presented in chapter\ref{chapter:implementation}. The document finally concludes
with a short summary where future work is discussed. 


