\section{Introduction}
The project is based on the language 'Flow' [1]. In contrast to the previous 
version of the project, it is built using Xtext 2.0 as language toolkit and 
Maven/Tycho for the build process. The target platform of the project is 
Eclipse 3.7. The project is split in several plugins and is structured as follows:
\begin{itemize}
  \item \textbf{de.hs\_rm.cs.vs.dsm.flow} \\
  The plug-in specifies the grammar of the language, provides a generator for 
  the language as well as scoping and naming mechanisms for the language editor.
  \item \textbf{ de.hs\_rm.cs.vs.dsm.flow.tests} \\
  Unit tests related to the language are placed in the plug-in and executed on 
  command line by the corresponding maven command.
  \item \textbf{de.hs\_rm.cs.vs.dsm.flow.ui} \\
  Within the plug-in different user-interface related classes and interfaces are
  specified. For example, the class which is registered as application extension 
  point for the Eclipse RCP application is defined within this plug-in.
  \item \textbf{de.hs\_rm.cs.vs.dsm.flow.feature} \\
  The plugin contains a feature.xml file which is required for the Eclipse RCP 
  application build. The file lists required and provided Eclipse plugins and 
  specifies licences, project website and other information.
  \item \textbf{de.hs\_rm.cs.vs.dsm.flow.product} \\
  Every Eclipse RCP application requires a product definition. The plugin 
  contains the product definition which defines
  \item \textbf{de.hs\_rm.cs.vs.dsm.owl} \\
  The plug-in consists of classes and interfaces generated from a RDFS and OWL 
  meta model.
  \item \textbf{de.hs\_rm.cs.vs.dsm.owl.edit} \\
  The plug-in offers different factory classes for creating OWL and RDFS 
  elements.
  \item \textbf{de.hs\_rm.cs.vs.dsm.owl.editor} \\
  Within the plug-in a editor for reading OWL files in <TODO> format is 
  provided. The plug-in only offers a simple tree-based editor since
  \item \textbf{de.hs\_rm.cs.vs.dsm.owl.ui} \\
  A plug-in which provides classes and extension points for loading the OWL 
  editor within the language editor. 
\end{itemize}
