\section{Summary}
The goal of this work was to provide a generator for the LUA interface of the
DSM framework. Future improvements and new research directions described in  
\cite{297:Frey2010} remain still open. This includes among others the definition
of new operators or the integration of knowledge tools in the RCP application. 

The Xtext language toolkit provides mechanisms for the creation of DSLs
and corresponding workbenches. This includes among others content assists, 
template proposals and quick fixes. Future developments of the query language 
and the language workbench might provide these features. However, the toolkit 
has its limitations, especially in terms of performance. The integration of 
multiple languages is considered an another important open issue. The query
language integrates different knowledge representations and corresponding 
rule/constraint languages. For example, the query language supports OWL and the
corresponding rule language SWRL. However, the workbench does not offer the same
features as a workbench for SWRL. While the current support of these additional
languages for the user is cumbersome, future version of the query language and 
the language workbench should offer more features such as content assists and 
code completion.

From a scientifc perspective questions about how to optimize the generated code
and operators remain open. 
